\documentclass[landscape,a0paper,fontscale=0.285]{baposter}

\usepackage{graphicx}
\graphicspath{{figures/}}

\usepackage{amsmath}
\usepackage{amssymb}

\usepackage{booktabs}
\usepackage{enumitem}
\usepackage{palatino}
\usepackage[font=small,labelfont=bf]{caption}

\usepackage{multicol}
\setlength{\columnsep}{1.5em}
\setlength{\columnseprule}{0mm}

\usepackage{tikz}
\usetikzlibrary{shapes,arrows}

\newcommand{\compresslist}{
\setlength{\itemsep}{1pt}
\setlength{\parskip}{0pt}
\setlength{\parsep}{0pt}
}

\begin{document}

\begin{poster}
{
  %% View grid for debugging.
  grid=false,
  %% Column specification.
  columns=3,
  colspacing=4em,
  %% Poster coloring, taken from WPI's poster template.
  bgColorOne=black!60!white!80!blue!98!green,
  bgColorTwo=black!95!white!95!blue!99!green,
  %% Border 
  borderColor=black,
  headerColorOne=white,
  headerColorTwo=white,
  headerFontColor=blue,
  boxColorOne=white,
  textborder=rounded,
  eyecatcher=true,
  headerborder=closed,
  headerheight=0.22\textheight,
  headershape=rounded,
  headershade=plain,
  headerfont=\huge\em, %Sans Serif
  textfont={\setlength{\parindent}{1.5em}},
  boxshade=plain,
  boxColorOne=white,
  borderColor=white,
  background=shadetb,
  linewidth=2pt
}
% University logo
{
  \includegraphics[width=0.2\linewidth]{wpi_logo.png}
} 
% Title
{\huge Software Similarity Detection With \textit{Checksims}\\[0.15em]
  \large Improvements to Algorithmic Comparisons and Usability}
% Authors
  {\large Theodore Meyer and Michael Andrews\\[0.2em]
  \{mandrews and tjmeyer\}@wpi.edu}
% Eye catcher
{
  \includegraphics[width=0.2\linewidth]{checksims_logo.png}
}

\headerbox{Abstract}{name=abstract, column=0, row=0, span=1}{
    In order to improve the quality and usability of the \textit{Checksims}
    software similarity detection tool, we present a series of modifications to
    the original project; changes include Syntax Tree comparison algorithms,
    integration with existing grading software, user interface improvements,
    and the ability to perform comparison against historical data.
}

\headerbox{Background}{name=background, column=0, row=0.23, span=1}{
  Academic dishonesty is often poorly defined. Different institutions,
  departments, professors, and courses have differing definitions and
  interpretations. The goal of \textit{Checksims} is to detect instances of
  potential unauthorized copying: similar sections of code that may constitute
  academic dishonesty.  Detecting similar sections of code is done in a number
  of ways. The two primary types of detection are token based comparison and
  syntax based comparison.  Token based detection algorithms work by splitting
  source text into smaller chunks and computing shortest modification distance.
  Syntax based detection parse code based on a defined grammar and generate a
  tree which an be operated on.  Our work on \textit{Checksims} has indicated
  that AST based methods are often more accurate than token based methods, but
  come with their own set of limitations.
}

\headerbox{Objectives}{name=objectives, column=0, row=0.620, span=1}{
  \begin{itemize}
  \item Design and implement an AST similarity detection algorithm, and then
    compare it to the Smith-Waterman in both accuracy and performance.
  \item Design a format and archiving system for comparing submissions not only
    to one another, but to previous course submissions as well.
  \item Integrate \textit{Checksims} with \textit{turnin}, the most commonly
    used tool for computer science assignment submissions at WPI\@.
  \item Create a graphical user interface for \textit{Checksims}, as currently
    it requires use of a command line interface.
  \end{itemize}
}


\headerbox{Parsing Process}{name=process, column=1, row=0.0, span=2}{
    
}


\headerbox{Results}{name=results, column=1, row=0.25, span=1}{
  Results
}

\headerbox{Analyses}{name=analyses, column=1, row=0.50, span=1}{
  Analyses
}

\headerbox{Acknowledgments}{name=acknowledgments, column=2, row=0.5, span=1}{
  ACK
}

\headerbox{References}{name=references, column=2, row=0.5, span=1}{
  Background: Lorem
}

\end{poster}

\end{document}
